\documentclass[a4paper]{tufte-handout}
\usepackage{tikz}
\usetikzlibrary{shapes,arrows}
\usepackage{hyperref}

\title{OpenSesame workshop 2019}% handout \& workbook}
\date{Nov 6th 5pm -- 7pm \& Nov 7th 9am -- 11am in P105}
\author{Matt Green \href{mailto:mgreen@bournemouth.ac.uk}{mgreen@bournemouth.ac.uk} }
%\author{Matt Green mgreen@bournemouth.ac.uk}
%\date{https://github.com/mjgreen/opensesame-workshop/archive/master.zip}

%\geometry{showframe} % display margins for debugging page layout

\usepackage{graphicx} % allow embedded images
  \setkeys{Gin}{width=\linewidth,totalheight=\textheight,keepaspectratio}
  \graphicspath{{graphics/}} % set of paths to search for images
\usepackage{amsmath}  % extended mathematics
\usepackage{booktabs} % book-quality tables
\usepackage{units}    % non-stacked fractions and better unit spacing
\usepackage{multicol} % multiple column layout facilities
\usepackage{lipsum}   % filler text
\usepackage{fancyvrb} % extended verbatim environments
  \fvset{fontsize=\normalsize}% default font size for fancy-verbatim environments

% Standardize command font styles and environments
\newcommand{\doccmd}[1]{\texttt{\textbackslash#1}}% command name -- adds backslash automatically
\newcommand{\docopt}[1]{\ensuremath{\langle}\textrm{\textit{#1}}\ensuremath{\rangle}}% optional command argument
\newcommand{\docarg}[1]{\textrm{\textit{#1}}}% (required) command argument
\newcommand{\docenv}[1]{\textsf{#1}}% environment name
\newcommand{\docpkg}[1]{\texttt{#1}}% package name
\newcommand{\doccls}[1]{\texttt{#1}}% document class name
\newcommand{\docclsopt}[1]{\texttt{#1}}% document class option name
%\newenvironment{docspec}{\begin{quote}\noindent}{\end{quote}}% command specification environment

\begin{document}
\bibliographystyle{plain}
\nobibliography{handout}
\maketitle
\marginnote{This handout, and all examples referred to in it, can be downloaded from \url{https://github.com/mjgreen/opensesame-workshop}}
\tableofcontents

\section{Getting Started}
\begin{enumerate}
\item \textbf{Download the workshop's supporting files:} these are  all the examples, and a copy of this handout: go to: \url{https://github.com/mjgreen/opensesame-workshop}. Look for the green button that says "Clone or Download". When you click it, it will offer you "Download ZIP": click on that, and when the zip file has downloaded, extract it somewhere you will be able to find (maybe the Desktop, or a folder in your home drive). If you don't know how to extract a zip file, right-click on the zip file and choose "extract all", then browse to the folder you want to use. It's important to get these files before we start, so if you need help, interrupt me and I will come and help you.
\item \textbf{Open up the OpenSesame program:} In P105, it is part of \emph{AppsAnywhere}. Be patient and only click the link once, otherwise you will get multiple instances of the program open up, which can be confusing. Again, stop me and get help if you have any trouble doing that, so that everyone has the program open before we go any further.
\end{enumerate}

\section{The simplest case of an experiment}
Every OpenSesame experiment is organised around a few core notions: \textbf{stimulus}, \textbf{instructions}, and \textbf{response}.

\subsection{Problem specification}
Let's imagine that we want to know whether the word \emph{war} is associated with positive or negative emotions.

\subsection{Solution in plain English}
We need to present the word (in this example the word \emph{war}) for some length of time (in this example it is 5 seconds); then instruct the participant what keys they can press (in this example it is \emph{n} for negative, or \emph{p} for positive); and then collect a response from the keyboard, only allowing \emph{n} for negative, or \emph{p} for positive).

\subsection{Solution as flowchart}
OpenSesame experiments can be visualised as flowcharts. Figure (\ref{simplestcase}) is the flowchart for this particular example, the simplest case of a stimulus-response experiment.
\begin{figure}
\centering
\tikzstyle{block} = [rectangle, draw, text width=15em, text centered, rounded corners, minimum height=4em]
\tikzstyle{line} = [draw, -latex']
\tikzstyle{cloud} = [draw, ellipse, node distance=2cm, minimum height=1em]
\begin{tikzpicture}[node distance = 2cm, auto]
\node [cloud] (init) {Start};
\node [block, below of=init] (stimulus) {\textbf{Stimulus:} present the word \emph{war} for 5 seconds};
\node [block, below of=stimulus] (instruct) {\textbf{Instructions:} say what the response keys are};
\node [block, below of=instruct] (collect) {\textbf{Response:} collect keyboard response, \emph{p} or \emph{n}};
\node [cloud, below of=collect] (stopping) {Stop};
\path [line] (init) -- (stimulus);
\path [line] (stimulus) -- (instruct);
\path [line] (instruct) -- (collect);
\path [line] (collect) -- (stopping);
\caption{Simplest stimulus-response experiment, presented as a flowchart}
\label{simplestcase}
\end{tikzpicture}
\end{figure}

%\FloatBarrier
\subsection{Solution as OpenSesame}
Go to your OpenSesame window, and do \texttt{File -> Open}, then browse to the folder where you extracted the suporting materials, and choose the following file: \texttt{example-01-the-simplest-case.osexp}

\marginnote{\textbf{.osexp} is the file extension for OpenSesame experiments, in the same way that \textbf{.jpeg} is the file extension for jpeg photos}


\end{document}

