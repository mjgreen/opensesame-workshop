\documentclass[a4paper]{tufte-handout}
\usepackage{tikz}
\usetikzlibrary{shapes,arrows}
\usepackage{hyperref}

\title{OpenSesame workshop handout \& workbook}
\date{Wednesday 6th November 5pm -- 7pm} % without \date command, current date is supplied
\author{Matt Green}

%\geometry{showframe} % display margins for debugging page layout

\usepackage{graphicx} % allow embedded images
  \setkeys{Gin}{width=\linewidth,totalheight=\textheight,keepaspectratio}
  \graphicspath{{graphics/}} % set of paths to search for images
\usepackage{amsmath}  % extended mathematics
\usepackage{booktabs} % book-quality tables
\usepackage{units}    % non-stacked fractions and better unit spacing
\usepackage{multicol} % multiple column layout facilities
\usepackage{lipsum}   % filler text
\usepackage{fancyvrb} % extended verbatim environments
  \fvset{fontsize=\normalsize}% default font size for fancy-verbatim environments

% Standardize command font styles and environments
\newcommand{\doccmd}[1]{\texttt{\textbackslash#1}}% command name -- adds backslash automatically
\newcommand{\docopt}[1]{\ensuremath{\langle}\textrm{\textit{#1}}\ensuremath{\rangle}}% optional command argument
\newcommand{\docarg}[1]{\textrm{\textit{#1}}}% (required) command argument
\newcommand{\docenv}[1]{\textsf{#1}}% environment name
\newcommand{\docpkg}[1]{\texttt{#1}}% package name
\newcommand{\doccls}[1]{\texttt{#1}}% document class name
\newcommand{\docclsopt}[1]{\texttt{#1}}% document class option name
%\newenvironment{docspec}{\begin{quote}\noindent}{\end{quote}}% command specification environment

\begin{document}
\bibliographystyle{plain}
\nobibliography{handout}
\maketitle
The materials for the workshop can be downloaded from this link: \href{https://github.com/mjgreen/opensesame_workshop_BU/archive/master.zip}
\tableofcontents

\section{The simplest case of an experiment}
Every OpenSesame experiment is organised around a few core notions: \textbf{stimulus}, \textbf{instructions}, and \textbf{response}.
%The simplest possible experiment presents one stimulus and collects one response.

\subsection{Problem specification}
Let's imagine that we want to know whether the word \emph{war} is associated with positive or negative emotions.

\subsection{Solution in plain English}
We need to present the word (in this example the word \emph{war}) for some length of time (in this example it is 5 seconds); then instruct the participant what keys they can press (in this example it is \emph{n} for negative, or \emph{p} for positive); and then collect a response from the keyboard, only allowing \emph{n} for negative, or \emph{p} for positive).

\subsection{Solution as flowchart}
OpenSesame experiments can be visualised as flowcharts. Figure (\ref{simplestcase}) is the flowchart for this particular example, the simplest case of a stimulus-response experiment.
\begin{figure}[!htbp]
\centering
\tikzstyle{block} = [rectangle, draw, text width=15em, text centered, rounded corners, minimum height=4em]
\tikzstyle{line} = [draw, -latex']
\tikzstyle{cloud} = [draw, ellipse, node distance=2cm, minimum height=1em]
\begin{tikzpicture}[node distance = 2cm, auto]
\node [cloud] (init) {Start};
\node [block, below of=init] (stimulus) {\textbf{Stimulus:} present the word \emph{war} for 5 seconds};
\node [block, below of=stimulus] (instruct) {\textbf{Instructions:} say what the response keys are};
\node [block, below of=instruct] (collect) {\textbf{Response:} collect keyboard response, \emph{p} or \emph{n}};
\node [cloud, below of=collect] (stopping) {Stop};
\path [line] (init) -- (stimulus);
\path [line] (stimulus) -- (instruct);
\path [line] (instruct) -- (collect);
\path [line] (collect) -- (stopping);
\caption{Simplest stimulus-response experiment, presented as a flowchart}
\label{simplestcase}
\end{tikzpicture}
\end{figure}

\FloatBarrier
\subsection{Solution as OpenSesame}
From now on I will use the shorthand \textbf{osexp} to mean "OpenSesame experiment": in the same way that the file extension for a jpeg photo is ".jpeg", the file extension for an opensesame experiment is ".osexp".

You can now download the first osexp that we will use in the workshop: \url{https://mjgreen.github.io/opensesame_workshop_BU/examples/example_01_the_simplest_case.osexp}

\clearpage
\section{Appendix}
\appendix

This is a minimal flowchart.

\begin{figure}
\centering
\begin{tikzpicture}[node distance = 2cm, auto]
\node [cloud] (init) {Start};
\node [cloud, below of=init] (stopping) {Stop};
\tikzstyle{line} = [draw, -latex']
\path [line] (init) -- (stopping);
\end{tikzpicture}
\end{figure}

This is a fully specified flowchart you can steal from.

\begin{figure}
\centering
% Define block styles
\tikzstyle{decision} = [diamond, draw, fill=blue!20, text width=5em, text badly centered, node distance=3cm, inner sep=0pt]
\tikzstyle{block} = [rectangle, draw, fill=blue!20, text width=10em, text centered, rounded corners, minimum height=4em]
\tikzstyle{line} = [draw, -latex']
\tikzstyle{cloud} = [draw, ellipse,fill=red!20, node distance=3cm, minimum height=1em]
     
\begin{tikzpicture}[node distance = 2cm, auto]
% Place nodes
\node [cloud, text width=5em, text badly centered] (init) {Start};
\node [block, below of=init] (assign) {i = 1};
\node [block, below of=assign] (increment) {increment i by 2};
\node [block, left of=increment, node distance=3cm, text width=5em, fill=red!20] (incBy9) {increment i by 9};
\node [block, below of=increment] (functionCall) {perform the function};
\node [decision, below of=functionCall] (evenOddCheck) {is i even or odd};
\node [cloud, below of=evenOddCheck] (stopping) {program closed then STOP};
 
% Draw edges
\path [line] (init) -- (assign);
\path [line] (assign) -- (increment);
\path [line] (increment) -- (functionCall);
\path [line] (functionCall) -- (evenOddCheck);
\path [line] (evenOddCheck) -| node [near start] {No} (incBy9);
\path [line] (incBy9) |- (assign);
\path [line] (evenOddCheck) |- (stopping);
\path [line] (evenOddCheck) -- ++(2.8cm,0cm) |- node [near start] {Yes} (assign);
 
\end{tikzpicture}
\caption{Problem formulation and the process} \label{fig:M1}
\end{figure}

\end{document}

