\documentclass[a4paper]{tufte-handout}

\title{OpenSesame workshop handout \& workbook}
\date{ Wednesday 6th November 5pm -- 7pm} % without \date command, current date is supplied
\author{Matt Green}

%\geometry{showframe} % display margins for debugging page layout

\usepackage{graphicx} % allow embedded images
  \setkeys{Gin}{width=\linewidth,totalheight=\textheight,keepaspectratio}
  \graphicspath{{graphics/}} % set of paths to search for images
\usepackage{amsmath}  % extended mathematics
\usepackage{booktabs} % book-quality tables
\usepackage{units}    % non-stacked fractions and better unit spacing
\usepackage{multicol} % multiple column layout facilities
\usepackage{lipsum}   % filler text
\usepackage{fancyvrb} % extended verbatim environments
  \fvset{fontsize=\normalsize}% default font size for fancy-verbatim environments

% Standardize command font styles and environments
\newcommand{\doccmd}[1]{\texttt{\textbackslash#1}}% command name -- adds backslash automatically
\newcommand{\docopt}[1]{\ensuremath{\langle}\textrm{\textit{#1}}\ensuremath{\rangle}}% optional command argument
\newcommand{\docarg}[1]{\textrm{\textit{#1}}}% (required) command argument
\newcommand{\docenv}[1]{\textsf{#1}}% environment name
\newcommand{\docpkg}[1]{\texttt{#1}}% package name
\newcommand{\doccls}[1]{\texttt{#1}}% document class name
\newcommand{\docclsopt}[1]{\texttt{#1}}% document class option name
%\newenvironment{docspec}{\begin{quote}\noindent}{\end{quote}}% command specification environment

\begin{document}

\bibliographystyle{plain}
\nobibliography{handout}

\maketitle% this prints the handout title, author, and date

\begin{abstract}
\noindent
\textbf{Intended Learning Outcomes:} at the end of this workshop you should have acquired basic competence with using OpenSesame, and you should be able to use the available OpenSesame-specific documentation, online tutorials, and online forums to build upon that basic competence.
I think that most beginners should be able to reach a point where they can create a simple stimulus-response experiment from scratch, having had no previous experience.
I think that most of those people would also be able to reach a point where they can take a more complicated template experiment, and adjust it until it produces the behaviour required for, say, a final year project experiment.
\end{abstract}

\tableofcontents

\section{Timeline}
\label{sec:timeline}

\begin{table}[ht]
  \centering
  \fontfamily{ppl}\selectfont
  \begin{tabular}{lcll}
    \toprule
    Time & Activity & Duration & \\
    \midrule
    17:00 -- 17:30 & Part 1 & 30 mins &       \\
    17:30 -- 17:40 & break  & 10 mins &  \\
    \addlinespace
    17:40 -- 18:20 & Part 2 & 40 mins &       \\
    18:20 -- 18:30 & break  & 10 mins & \\
    \addlinespace
    18:30 -- 19:00 & Part 3 & 30 mins &       \\
    \bottomrule
  \end{tabular}
  \caption{Timeline}
  \label{tab:timeline}
%  \zsavepos{pos:normaltab}
\end{table}


\end{document}

